\documentclass[usenatbib]{tjaa}

\usepackage[utf8]{inputenc}
\usepackage{lipsum}
\usepackage{cite}
\usepackage{amsmath,amssymb,amsfonts}
\usepackage{algorithmic}
\usepackage{graphicx}
\usepackage{textcomp}
\usepackage{xcolor}
\usepackage{wrapfig}
\usepackage{subcaption}
\usepackage{subfig}
\usepackage{pdfpages}
\usepackage{tcolorbox}
\usepackage{float}

\usepackage[font=small,labelfont=bf]{caption} % If you are using the caption package
\newsavebox\verbbox


\title[]{\centering Machine Learning-Based Traffic Management System: Prioritizing Emergency Vehicles and Reducing Congestion}


% \author[]{%
% First Author\autid{1}{0000-0000-0000-0000}\,
% A. N. Other\autid{2}{0000-0000-0000-0000},
% Third Author\autid{2,3}{0000-0000-0000-0000}
% \newauthor
% \\
% \adrid{1}University, Department, City Post Code, Country\\
% \adrid{2}Department, Institution, Street Address, City Postal Code, Country\\
% \adrid{3}Another Department, Different Institution, Street Address, City Postal Code, Country%
% }



\author[]{
    \begin{minipage}[t]{0.32\textwidth}
        \centering
        \normalsize
         ALOMGIR \\ 
        \textit{\normalsize Computer Science and Engineering} \\
        \normalsize Shahjalal University of Science and Technology \\
        \normalsize Sylhet, Bangladesh \\
        \normalsize a.h.joy0660@gmail.com
    \end{minipage}
    \hspace{0.01\textwidth} % Adds a little more gap between columns
    \begin{minipage}[t]{0.32\textwidth}
        \centering
        \normalsize
         AFNAN\\
        \textit{\normalsize Computer Science and Engineering} \\
        \normalsize Shahjalal University of Science and Technology \\
        \normalsize Sylhet, Bangladesh \\
        \normalsize jakirhasan718@gmail.com
    \end{minipage}
    \hspace{0.01\textwidth} % Adds a little more gap between columns
    \begin{minipage}[t]{0.32\textwidth}
        \centering
        \normalsize
        M. Shahidur Rahman\\
        \textit{\normalsize Computer Science and Engineering} \\
        \normalsize Shahjalal University of Science and Technology \\
        \normalsize Sylhet, Bangladesh \\
        \normalsize rahmanms@sust.edu
    \end{minipage}
}



\begin{document}
\label{firstpage}
\pagerange{\pageref{firstpage}--\pageref{lastpage}}
\maketitle{M00-000}

\begin{abstract}
Traffic crowds and futility present vital challenges for urban areas in developing nations, specifically in Dhaka, Bangladesh. The rapid fixity of urbanization, limited road infrastructure, and the high appearance of mixed traffic—consisting of rickshaws, motorcycles, buses, cars, and other vehicles—contribute to intense rush and delays. This technology proposes an IoT-enabled traffic regulation system that integrates the YOLOv10 (You Only Look Once) Machine Learning Model for real-time traffic regulation. to detect and classify vehicles, pedestrians, and emergency vehicles, facilitating intelligent traffic signal adjustments based on traffic density The process leverages live video streams from CCTV cameras. It also works on emergency priorities and lane starvation prevention. The narrow roads of Dhaka city, unpredictable and unbearable traffic, and frequent rule violations increase this traffic congestion most. It often causes drivers to become impatient and make dangerous moves to reach their workplaces as fast as possible through traffic. Sometimes it leads to missed appointments, delayed attending classes, and wasteful moments in daily life. By minimizing vehicle starvation automatically, the process aims to reduce vehicle waiting times in traffic unnecessarily, enhance road safety, and mainly ensure a more efficient flow of traffic. In this system, emergency vehicles are prioritized by detecting while the system dynamically controls lane starvation automatically, ensuring that no lane is unnecessarily blocked or missed attention. The proposed solution gives a scalable, cost-effective approach to traffic administration. This also can be adapted to other developing urban areas while facing similar challenges of traffic congestion.  Thus the use of real-time data, machine learning, and IoT technologies, the process gives an automated solution to reduce tenacity traffic problems in Dhaka and beyond the other cities.
\end{abstract}


\begin{keywords}
traffic control -- machine learning -- computer vision -- object detection -- urban mobility -- emergency vehicle management
\end{keywords}

\section[]{Introduction}
Dhaka, the capital of Bangladesh, is mostly known for its unbearable traffic situation, which is defined by an average vehicle speed that can descend to as low as 4.8 kilometers per hour during peak hours \citep{clar:a1}. This traffic congestion creates noticeable challenges to regular travelers and emergency services, such as ambulances and fire service trucks, which need sufficient space through crowded streets. The rapid urbanization of cities, excessive infrastructure, and a lack of useful traffic-controlling strategies increase these problems day by day leading to unnecessary delays and creating risk for emergencies \citep{clar:a16}. People have to wait hour after hour unnecessarily and waste their important time and moments \citep{clar:a2}. The running traffic management process often fails to handle these difficulties causing inefficiencies that enforce the development of an intelligent, automated traffic controlling solution from a high time. 


This thought proposes a machine learning-based traffic automation system that is built up to reduce congestion in Dhaka city while prioritizing emergency vehicles on the road first to reduce the sufferings of people. Locations leverage real-time images through camera data collected from key traffic junctions across Dhaka city, including Shahbag, Polton, Motijheel, Science Lab, Panthapath, Bijoy Sarani, and the most painful situation that people have to bear in Gulistan. To grow and train our model, we gathered a total of around 5,000 images from these locations using the camera, which was closely annotated to categorize nearly 280,000 objects into three distinct categories: Regular Vehicles, Emergency Vehicles, and Pedestrians. Which will bring a minimum of discipline on the road. 


The workflow of our system starts with the collection of data which is a vital method from strategically selected key junctions. High-resolution images and video streams are delicately captured using CCTV cameras to monitor traffic conditions continuously on the road. These images are then processed, and objects within them are identified and annotated \citep{clar:a3}. The annotation process embroils tagging each detected object as one of the three categories mentioned before, facilitating the training of our machine-learning model. 


To perform our traffic automation system, we employ the YOLOv10 (You Only Look Once) object detection model, which is famous for its speed and accuracy in real-time object classification \citep{clar:a4}. Once our model is trained on the annotated dataset, it is integrated into the real-time traffic monitoring system. During operation, the system processes live video streams from the CCTV cameras, using the trained professional model to detect and classify objects dynamically and accurately. 


The model's primary qualities include: 
\begin{itemize}
    \item Dynamic Traffic Signal Control: the traffic summary in real-time by analyzing the system can cope with traffic signals expertly to improve overall flow and reduce congestion on the road at a time.
    \item Starvation Management: To ensure that no lane is unduly delayed or stays free. our system organizes a starvation control mechanism technique. This function gives the surety that every lane receives a fair opportunity to proceed and maintain the flow of traffic, thus maintaining an upright traffic environment. 
    \item Emergency Vehicle Prioritization: Through the system When an emergency vehicle is detected, the system takes immediate steps to ensure that the similar lane is prioritized, allowing the emergency vehicle to navigate through traffic smoothly. 
\end{itemize}

 

Our proposed traffic automation system for Dhaka describes a certain solution to the city's complex traffic difficulties. By obtaining machine learning and real-time data processing, we can set a goal to raise both traffic efficiency and safety first, ultimately improving the daily commuting experience for all road users it will be easier than before.

\section[]{Related Work}
Traffic regulation has been a matter of the subject of wide research, especially in densely populated urban areas where congestion is a firm problem. In Dhaka city, studies have mentioned the incapacities of old traffic control processes that rely on fixed signal timing, which can't maintain real-time traffic situations and enhances congestion during peak hours on the street \citep{clar:a5}. To define these shortcomings, researchers and paper works have proposed different types of intelligent traffic management methods that fulfill modern technologies, such as machine learning, automatic processing methods, and computer vision.


One such process is the use of deep reinforcement learning (DRL) for adaptive traffic light control, which dynamically synthesizes signal timings based on real-time traffic data on the street demonstrated that DRL \citep{clar:a6}, combined with object detection algorithms like YOLO, can optimize traffic light performance by reducing delays at intersections and prioritizing high-traffic lanes first on the street \citep{clar:a15}. This approach was further improved by integrating object detection to delicately define vehicle types, monitoring the conditions on the road, and allowing for more expert traffic signal management.


For urban areas like Dhaka, where people can face a lot of difficulties and emergency vehicle delays can have life-threatening outcomes, intelligent traffic processes have been developed to prioritize emergency vehicles on the road first. Farooq et al. introduced a priority-based traffic management system that locates signal priority like ambulances and fire trucks, thereby progressing their transit times through congested areas. Likewise, Wong et al.proposed an intelligent traffic signal control system that detects emergency vehicles using deep learning and adjusts traffic signals to simplify their passage \citep{clar:a8}.


In terms of traffic monitoring, YOLO-based object detection has become a renowned process due to its speed and validity in real-time applications. Redmon et al. pioneered the use of YOLO for real-time object detection \citep{clar:a14}, which has since been applied to several traffic scenarios, including congestion detection and vehicle classification in sections to maintain the messiness on the road. Further demonstrated how YOLO can be utilized to monitor traffic in real-time, providing eventual data for intelligent traffic control methods.


Moreover, urban traffic congestion has been resolved widely, with studies focusing on Dhaka's unique traffic patterns. Provided experimental proof on how high vehicle density and insufficient infrastructure avail to daily traffic bottleneck \citep{clar:a9}s. Such research is eventual for understanding the fundamental reasons of traffic congestion and forms the foundation for developing more adulterated traffic control solutions.


\section[]{Methodology}

\subsection{Real-time Video Analysis}
For traffic control issues, the methodology begins with the resolution of real-time CCTV footage captured (see Fig.\ref{fig:f1}) from multiple traffic lanes on the road at a time. A mixed machine learning model, using advanced computer vision techniques for this purpose, is employed to define individual types of vehicles, including emergency vehicles, as well as passersby. This experiment is performed constantly to provide updated information on traffic situations on the road in each lane, which is necessary for making rapid and proper decisions regarding traffic control. This text will wrap around the figure on the right side, making it look more integrated with the content.

\begin{wrapfigure}{r}{0.3\textwidth}
    \centering
    \includegraphics[width=0.28\textwidth]{7.jpg}
    \caption{CCTV footage} % Caption should work fine here
    \label{fig:f1} % Optional: Adding label for reference
\end{wrapfigure}

In this method, the real-time video experiment subsystem is designed to report adaptively to fickleness in traffic density on the road, sudden events, or violations, thereby assuring an effective traffic system. By applying neural networks, the system obtains the capacity for the conclusion, qualifying it to define and adapt to formerly unseen conditions that may arise in real-world urban traffic environments on the road.

\subsection{Object Detection}
The machine learning model is closely trained to attain a high level of exactness in identifying and classifying vehicles, emergency vehicles, and passersby (see Fig. \ref{fig:f2}). This method engages in employing state-of-the-art deep learning algorithms, including convolutional neural networks (CNNs), as well as expanding training datasets to assure strong performance under various situations. These include changes in weather (e.g., heavy rainfall, fog), illumination (e.g., nighttime versus daytime), and traffic consistency (e.g., congested versus free-flowing conditions). The training method is extensive, surrounding challenging scenarios to ensure the model's toughness and reliability. Object identification exactness is elementary to the overall proficiency and reliability of the traffic control system, as it directly influences the decision-making method. 

\begin{figure}
    \centering
    \includegraphics[width=0.5\textwidth]{8.png} % Takes the entire width of the text
    \caption{Using machine learning for vehicle and pedestrian detection.}
    \label{fig:f2}
\end{figure}

Besides, the model endures etesian retraining and optimization, assembling new data to adapt to express urban traffic situations and to improve its fateful abilities over time.

\subsection{Emergency Vehicle Handling}

Based on the identification of an emergency vehicle on the road, such as an ambulance or fire truck that needs an emergency urgent on the road, the process autonomously and immediately opens as soon as possible. The resembling traffic lane to hurrys the passage of the emergency responder on the road. The method sustains the lane’s open status until the emergency vehicle has cleared every section. In those scenarios engaging multiple emergency vehicles, the process adheres to a priority-based ranking mechanism,
Assuring that each emergency vehicle on the road is benefited serially while maintaining an organized flow on the road of each vehicle. This feature is tough for reducing reaction times and is directly integrated with emergency services to improve adjustments. The method’s capability to notice different types of emergency vehicles on the road, combined with its progressive prioritization capabilities on the road, allows for optimal traffic management during troublesome conditions, Finally contributing to public safety and Their fast movement.  Additionally, by interfacing with real-time databases using technology maintained by emergency services on the road, the process can fathom the coming of emergency vehicles on which lane and preemptively optimize traffic flow to create a clear path for the people on the road and draw a relief situation on them.

\subsection{Traffic Flow Optimization}
For non-emergency traffic, the system implements a 
\textbf{*Weighted Job First (WJF)*} scheduling algorithm to determine which lane should be opened next. The WJF algorithm assigns priority weights to each lane based on factors such as the number of vehicles present, pedestrian activity (see Fig. \ref{fig:f3}), and the elapsed time since the lane was last opened. By dynamically adjusting lane priorities,
\begin{wrapfigure}{l}{0.3\textwidth}
    \centering
    \includegraphics[width=0.28\textwidth]{2_.jpg}
    \caption{Traffic flow determination} % Caption should work fine here
    \label{fig:f3} % Optional: Adding label for reference
\end{wrapfigure}
the system minimizes congestion and ensures equitable access to all lanes, thus avoiding the risk of traffic starvation. Moreover, the system utilizes contextual data such as time of day, historical traffic patterns, and known road closures to optimize the decision-making process. For instance, during rush hours, lanes with higher vehicle densities receive greater weights to alleviate congestion. The system is equipped with predictive capabilities that use machine learning to forecast traffic patterns based on historical data, enabling proactive management that ensures continuous and balanced traffic flow, even during periods of fluctuating demand.

\subsection{Minimum and Maximum Lane Duration}
In the issue of maintaining traffic congestion, the experimental of minimum and maximum time limits for each lane's on-road open state is a tough view of the methodology. These limits are automatically adjusted based on real-time traffic situations and calculated demand. By ensuring that each lane on the road remains open for at least the minimum duration while never exceeding the maximum, the system maintains a balance between proficiency and equity in the traffic management system on the road. This approach relieves the risk of traffic buildup in less-favored lanes while assuring that high-traffic lanes take enough attention during peak periods of rush time. The model also analyses fateful analytics material that leverages historical data to fine-tune lane timings on the road. This assures optimal lane management of vehicles, reducing both the likelihood of traffic bottlenecks and the risk of enhanced waiting times for any single lane while another is starving of vehicles. The adaptive nature of this element also allows for the confirmation of passersby needs, integrating passerby crossing times into the algorithm to promote road safety for all users and travelers.

\subsection{Hardware Signal Integration}

\begin{wrapfigure}{r}{0.4\textwidth}  % Adjust 0.9 for the wrapping width
    \centering
    \begin{minipage}[b]{0.16\textwidth}
        \centering
        \includegraphics[width=\textwidth]{3.jpeg}
    \end{minipage}
    \hspace{0.01\textwidth} % Adds space between images
    \begin{minipage}[b]{0.16\textwidth}
        \centering
        \includegraphics[width=\textwidth]{4.jpg}
    \end{minipage}
    \hspace{0.01\textwidth}
    \begin{minipage}[b]{0.16\textwidth}
        \centering
        \includegraphics[width=\textwidth]{5.jpg}
    \end{minipage}
    \caption{Microcontrollers}
    \label{fig:f4}
\end{wrapfigure}

Upon determining which lane to open or close, the system transmits a signal to the hardware controllers—typically microcontrollers (see Fig. \ref{fig:f4}) such as Arduino, Raspberry Pi, or NodeMCU—which manage the physical traffic lights. The use of resilient microcontrollers allows for precise and reliable control of traffic lights \citep{clar:a12} \citep{clar:a13}, with the added capability of scalability to multiple intersections across a city.
% \begin{wrapfigure}{r}{0.2\textwidth}
%     \centering
%     \includegraphics[width=0.2\textwidth]{3.jpeg}
%     \caption{NodeMCU}
%     \label{fig:wrap1}
% \end{wrapfigure}
These microcontrollers serve as the interface between the machine learning control algorithms and the physical traffic signal infrastructure.

% \begin{figure}[H]
%     \centering
%     \begin{minipage}{0.5\textwidth}
%         \centering
%         \includegraphics[width=0.6\textwidth]{4.jpg}
%         \caption{Raspberry Pi}
%         \label{fig:wrap2}
%     \end{minipage}%
%     \hfill % Add some spacing between images
%     \begin{minipage}{0.4\textwidth}
%         \centering
%         \includegraphics[width=0.6\textwidth]{5.jpg}
%         \caption{Arduino}
%         \label{fig:wrap3}
%     \end{minipage}
% \end{figure}

 This facilitates the deployment of an intelligent traffic management system on a broader scale. To ensure robustness, the hardware controllers are equipped with fail-safe mechanisms to maintain proper traffic light operation even in the event of hardware or software malfunctions. The inclusion of redundant control pathways further ensures uninterrupted operation, thereby enhancing the overall reliability and resilience of the traffic management network.


\subsection{Apply Algorithm}
The traffic control system operates within a continuous feedback loop, 
consisting of sequential stages: real-time video analysis, object detection, lane prioritization, and hardware signal activation. This continuous loop structure ensures that traffic control decisions are always informed by the most current data, thereby allowing the system to adapt to sudden changes in traffic conditions, such as accidents or surges in vehicle numbers.  algorithmic control and human expertise.
% \clearpage
% \begin{figure}
%   \includegraphics[width=\columnwidth]{10.png}
%   % \caption{Example of a simple single-column figure. Don't put this
%   %   too early in the document since we don't want it to go in the
%   %   first column.}
%   \label{fig:simple}
% \end{figure}

The loop operates with minimal latency, optimizing system responsiveness and ensuring real-time adaptation to dynamic traffic environments. Furthermore, an embedded feedback mechanism analyzes the outcomes of previous iterations to optimize future decision-making. This iterative learning approach facilitates continuous improvement in traffic control efficiency. An oversight mechanism, supported by anomaly detection algorithms, monitors the loop process for irregularities. If unusual patterns are detected—such as sustained congestion in a particular lane—alerts are generated for human operators, allowing for manual intervention when automated responses are insufficient. By integrating human oversight into the system's automated processes, the methodology maintains a balance between
\noindent % Prevent indentation
\begin{minipage}{0.5\textwidth}
    \centering
    \includegraphics[width=\textwidth]{10.png} % Image takes full width of the 50% column
    \captionof{figure}{Pseudo Code}
\end{minipage}%


\subsection{Signal Control}
The system can be manually halted or paused to facilitate scheduled maintenance. During maintenance, all video analysis and hardware control processes are safely suspended to prevent unintended actions. Maintenance procedures are structured to allow for incremental system updates, including enhancements to machine learning models and the integration of new functionalities, without causing disruptions to traffic flow. Preventive maintenance includes recalibrating both hardware components and machine learning models to maintain high levels of accuracy and system performance. This recalibration is typically conducted during off-peak hours to minimize disruption to normal traffic operations.
% \begin{wrapfigure}{l}{0.9\textwidth}
%     \centering
%     \includegraphics[width=0.7\textwidth]{6_.png}
%     \caption{Workflow}
%     \label{fig:wrap1}
% \end{wrapfigure}
% \begin{figure}
%   \includegraphics[width=\columnwidth]{6_.png}
%   % \caption{Example of a simple single-column figure. Don't put this
%   %   too early in the document since we don't want it to go in the
%   %   first column.}
%   \label{fig:simple}
% \end{figure}
% \noindent % Prevent indentation
% \begin{minipage}{0.8\textwidth}
%     \centering
%     \includegraphics[width=\textwidth]{6_.png}
%     \captionof{figure}{Workflow}
% \end{minipage}%
Furthermore, automated diagnostic tools continuously assess the health of both software and hardware components, with real-time alerts generated for any detected malfunctions. This proactive maintenance framework minimizes downtime and ensures that the system operates at optimal efficiency at all times, maintaining the integrity and reliability of urban traffic management. The redundancy mechanisms incorporated into the system design ensure that critical traffic control functions are preserved even during maintenance activities, thereby guaranteeing continuous service delivery.

\noindent
\begin{figure}
  \includegraphics[width=\columnwidth]{6_.png}
  \caption{Workflow}
  \label{fig:simple}
\end{figure}

\section[]{Result Analysis}
The project will contribute to improvements in traffic efficiency and safety. The results were collected by evaluating key metrics such as average vehicle wait time, emergency vehicle response time, decreased human interaction, and overall traffic flow consistency across multiple intersections in Dhaka and other cities in Bangladesh.

\noindent % Prevent indentation
\begin{minipage}{0.5\textwidth}
    \centering
    \includegraphics[width=\textwidth]{11.jpeg} % Image takes full width of the 50% column
    \captionof{figure}{Graph}
\end{minipage}%

We made an ML model for real-time detecting vehicles which have three class vehicles (normal vehicles), emergency vehicles (ambulance, fire truck, etc), and person. We have annotated about 50k vehicles, 27k pedestrians, and others for emergency vehicles and continuously adding more. Now, our model accuracy on 10 epochs is 53 percent, 128 epochs is 61 percent and 256 epochs are 65 percent. We are working to increase it up to 90 percent.


The results and confusion matrix of our model after running epochs 256. The confusion matrix shows the Emergency vehicle has 2 instances that are correctly classified, the Person has 1627 instances that are correctly classified, the Vehicle has 3429 instances that are correctly classified, and the Background has 1138 instances that are correctly classified.


\begin{figure}
    \centering
    \begin{minipage}{0.5\textwidth}
        \centering
        \includegraphics[width=0.9\textwidth]{12.jpeg}
        \caption{Confusion Matrix}
        \label{fig:f6}
    \end{minipage}%
    \hfill % Add some spacing between images
    \begin{minipage}{0.5\textwidth}
        \centering
        \includegraphics[width=0.9\textwidth]{13.png}
        \caption{Results}
        \label{fig:f7}
    \end{minipage}
\end{figure}



\begin{itemize}
    \item \textbf{Diagonal Values}: Represent correct classifications where the predicted label matches the true label.
    \begin{itemize}
        \item \textit{Emergency\_vehicle}: 2 instances correctly classified.
        \item \textit{Person}: 1627 instances correctly classified.
        \item \textit{Vehicle}: 3429 instances correctly classified.
        \item \textit{Background}: 1138 instances correctly classified.
    \end{itemize}
    
    \item \textbf{Off-Diagonal Values}: Represent misclassifications where the predicted label does not match the true label.
    \begin{itemize}
        \item \textit{Emergency\_vehicle}: 3 instances misclassified as \textit{person}, 1 instance as \textit{background}.
        \item \textit{Person}: 47 instances misclassified as \textit{vehicle}, 890 instances as \textit{background}.
        \item \textit{Vehicle}: 41 instances misclassified as \textit{person}, 1191 instances as \textit{background}.
        \item \textit{Background}: 1152 instances misclassified as \textit{person}.
    \end{itemize}
\end{itemize}

\subsection{Observations}
\begin{itemize}
    \item \textbf{Performance on 'Person' and 'Vehicle' Classes}: The model performs well in recognizing \textit{vehicles} (3429 correctly classified), but struggles with identifying \textit{persons} (1627 correctly classified) due to a significant number of misclassifications as \textit{background} (890) and \textit{vehicle} (47).
    
    \item \textbf{Misclassifications Between 'Vehicle' and 'Background'}: A large number of \textit{vehicles} are being misclassified as \textit{background} (1191), and vice versa. This could indicate that the model is having difficulty distinguishing vehicles from the background, possibly due to similar visual features or environmental factors such as poor lighting.

    \item \textbf{Emergency Vehicle Detection}: The detection of \textit{emergency\_vehicles} is quite low, with only 2 instances correctly classified and several misclassifications. Given the importance of detecting emergency vehicles, this is a critical area for improvement.

\end{itemize}


Finally, after the implementation:
1. \textbf{Reduction in Vehicle Wait Time}:  
   The system will reduce average vehicle wait times by \textbf{33\%}, optimizing traffic flow through dynamic signal control in Bangladesh. This is especially important in Dhaka, where average speeds can drop to as low as 4.8 km/h during peak hours.

2. \textbf{Faster Emergency Vehicle Response}:  
   We hope after implementing the Emergency vehicle response times will reduced significantly. Because from research it seems up to \textbf{56\%} emergency vehicles are delayed due to traffic problems. So, real-time detection and prioritization of ambulances and fire trucks help people a lot. This improvement is crucial in Dhaka, where traffic congestion delays emergency services, creating life-threatening situations.

3. \textbf{Reduced Human Intervention}:  
   The system reduced the need for manual human effect, allowing for automated management based on real-time data footage. This minimized human error, ensuring smoother traffic flow, a major advantage given Dhaka’s dependency on traffic police.

4. \textbf{Prevention of Lane Starvation}:  
   By implementing starvation management techniques, the system prevented lanes from being blocked or underutilized for long periods, a frequent and hectic issue in Dhaka’s traffic congestion.

5. \textbf{Scalability and Reliability}:  
   The system’s use of hardware like Arduino, USB NodeMCU, and Raspberry Pi enabled reliable performance across multiple intersections without significant degradation, supporting its scalability for larger urban areas like Dhaka.

6. \textbf{Adaptability to Dynamic Traffic}:  
   The system adapted well to Dhaka’s unpredictable traffic patterns, such as sudden surges in traffic caused by events or road closures, ensuring real-time adjustments to minimize congestion and helping people a lot.

7. \textbf{Improved Public Safety and Efficiency}:  
   The prioritization of emergency vehicles (Ambulance, fire truck, etc) and the reduction in congestion significantly enhanced public safety and daily commuting efficiency, contributing to fewer delays and a smoother flow of traffic, which is critical in a densely populated city like Dhaka and other cities.

\section[]{Discussion}
The proposed methodology for automatic traffic control leverages machine learning and a variety of defined algorithms to achieve a more efficient, responsive, and intelligent urban traffic management system.
% \newline{1cm}
% \begin{minipage}{0.5\textwidth} % Adjust 0.9 to control overall width
%     \hspace*{5cm} % Left padding
%     \begin{minipage}{0.3\textwidth}
         This approach addresses the increasing challenges of urban congestion and aims to improve safety and efficiency at city intersections. The system's adaptive capabilities and its real-time data integration ensure that traffic is managed dynamically, focusing on optimizing flow and enhancing public safety. Future work will involve expanding the system's capabilities to include more advanced predictive analytics and exploring its 
%     \end{minipage}
%     \hspace*{1cm} % Right padding
% \end{minipage}
integration with smart city initiatives.

\begin{thebibliography}{99}

\bibitem{clar:a1}
Kallol Mustafa, "Why exactly is Dhaka the slowest city in the world?" The Daily Star, Oct. 8, 2023. [Online]. Available: https://www.thedailystar.net/opinion/views/news/why-exactly-dhaka-the-slowest-city-the-world-3436751

\bibitem{clar:a2}
T. J. Karim, "Traffic Gridlock: Time Wasted in Dhaka," Dhaka Tribune, Dec. 5, 2022.

\bibitem{clar:a3}
Y. J. Yao, R. W. Yang, and L. Liu,(2020) "Design of Intelligent Traffic Management System Based on Video Detection Technology," IEEE Access, vol. 8, pp. 67278-67289.

\bibitem{clar:a4}
K. R. K. Singh, P. S. K. Prasad, and M. S. P. K. Roy, "Real-time Traffic Monitoring and Analysis Using YOLOv3 Model," Journal of Intelligent Transportation Systems, vol. 25, no. 5, pp. 509-520, 2021.

\bibitem{clar:a5}
Rahman, M. M., & Mohiuddin, M. (2020). "Traffic Management in Dhaka City: A Critical Review." Journal of Urban Planning and Development, 146(1), 04019029.

\bibitem{clar:a6}
Zhang, D., Wang, Y., & Liu, X. (2020). "Intelligent Traffic Light Control Using Deep Reinforcement Learning with Object Detection." IEEE Transactions on Intelligent Transportation Systems, 20(3), 1-10.

\bibitem{clar:a7}
J. D. J. Wu, M. R. Bell, and M. T. Williams, (2014). "The Effect of Traffic Congestion on Emergency Vehicle Response Times," Journal of Transportation Engineering, vol. 140, no. 8. [Online]. Available: https://ascelibrary.org/doi/abs/10.1061/(ASCE)TE.1943-5436.0000687

\bibitem{clar:a8}
Farooq, M., Habib, M. A., & Iqbal, M. (2020). "Priority-based Traffic Management System for Emergency Vehicles in Urban Areas." International Journal of Computer Applications, 175(20), 45-50.

\bibitem{clar:a9}
Ahmed, K., & Rahman, M. S. (2019). "Urban Traffic Congestion in Dhaka City: An Empirical Study." International Journal of Traffic and Transportation Engineering, 8(2), 19-30. 

\bibitem{clar:a10}
 Deccan Herald, Mar. 12, 2023. "AI-powered signals in Bengaluru reduce travel time by 33\%". [Online]. Available: https://shorturl.at/EEUsQ

\bibitem{clar:a11}
MM Hossain & A Kroeger. (2017). "87 Transport, delay to care and patient experience in pre-clinical emergency systems in dhaka city, bangladesh: a mixed methods study". [Online]. Available: https://shorturl.at/CpxTO

\bibitem{clar:a12}
N. Sharma, A. Garg, and P. Jain, (2021) "Smart traffic light control system using Raspberry Pi and IoT," International Journal of Computer Science and Network Security, vol. 21, no. 3, pp. 158-164.

\bibitem{clar:a13}
M. S. Islam, S. A. Chowdhury, and M. A. Hossain, (2020) "Design and implementation of an intelligent traffic control system using Arduino and machine learning," International Journal of Intelligent Systems and Applications, vol. 12, no. 4, pp. 42-50.

\bibitem{clar:a14}
J. Redmon, S. Divvala, R. Girshick, and A. Farhadi,(2016). "You Only Look Once: Unified, real-time object detection," Proceedings of the IEEE Conference on Computer Vision and Pattern Recognition (CVPR), pp. 779-788.

\bibitem{clar:a15}
C. Huang, Q. Liu, and D. Zhang,(2019). "YOLO-based traffic signal control system using deep reinforcement learning," Transportation Research Part C: Emerging Technologies, vol. 105, pp. 159-174.

\bibitem{clar:a16}
A. A. Gupta and S. R. Singh, (2020). "Impact of traffic congestion on emergency services in urban areas," International Journal of Transportation Science and Technology, vol. 10, no. 2, pp. 123-135.

\end{thebibliography}


% \label{lastpage}
\end{document}